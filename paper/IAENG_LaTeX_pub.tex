
%% IAENG_pub.tex 2010/08/30
%% It is based on the bare_jrnl.tex V1.3 2007/01/11 by Michael Shell
%% see http://www.michaelshell.org/
%% for current contact information.
%%
%% This is a skeleton file demonstrating the use of IAENGtran.cls
%% (requires IAENGtran.cls version 1.7 or later) with an IAENG journal/conference paper.
%%
%% Support sites:
%% http://www.michaelshell.org/tex
%% http://www.ctan.org/tex-archive/macros/latex/contrib/IEEEtran/

% *** Authors should verify (and, if needed, correct) their LaTeX system  ***
% *** with the testflow diagnostic prior to trusting their LaTeX platform ***
% *** with production work. IAENG's font choices can trigger bugs that do  ***
% *** not appear when using other class files.                            ***
% The testflow support page is at:
% http://www.michaelshell.org/tex/testflow/


%%*************************************************************************
%% Legal Notice:
%% This code is offered as-is without any warranty either expressed or
%% implied; without even the implied warranty of MERCHANTABILITY or
%% FITNESS FOR A PARTICULAR PURPOSE!
%% User assumes all risk.
%% In no event shall IAENG or any contributor to this code be liable for
%% any damages or losses, including, but not limited to, incidental,
%% consequential, or any other damages, resulting from the use or misuse
%% of any information contained here.
%%
%% All comments are the opinions of their respective authors and are not
%% necessarily endorsed by the IAENG.
%%
%% This work is distributed under the LaTeX Project Public License (LPPL)
%% ( http://www.latex-project.org/ ) version 1.3, and may be freely used,
%% distributed and modified. A copy of the LPPL, version 1.3, is included
%% in the base LaTeX documentation of all distributions of LaTeX released
%% 2003/12/01 or later.
%% Retain all contribution notices and credits.
%% ** Modified files should be clearly indicated as such, including  **
%% ** renaming them and changing author support contact information. **
%%
%%*************************************************************************

% Note that the a4paper option is mainly intended so that authors in
% countries using A4 can easily print to A4 and see how their papers will
% look in print - the typesetting of the document will not typically be
% affected with changes in paper size (but the bottom and side margins will).
% Use the testflow package mentioned above to verify correct handling of
% both paper sizes by the user's LaTeX system.
%
% Also note that the "draftcls" or "draftclsnofoot", not "draft", option
% should be used if it is desired that the figures are to be displayed in
% draft mode.
%
\documentclass[journal]{IAENGtran}
%
% If IAENGtran.cls has not been installed into the LaTeX system files,
% manually specify the path to it like:
% \documentclass[journal]{../sty/IAENGtran}





% Some very useful LaTeX packages include:
% (uncomment the ones you want to load)


% *** MISC UTILITY PACKAGES ***
%
%\usepackage{ifpdf}
% Heiko Oberdiek's ifpdf.sty is very useful if you need conditional
% compilation based on whether the output is pdf or dvi.
% usage:
% \ifpdf
%   % pdf code
% \else
%   % dvi code
% \fi
% The latest version of ifpdf.sty can be obtained from:
% http://www.ctan.org/tex-archive/macros/latex/contrib/oberdiek/
% Also, note that IAENGtran.cls V1.7 and later provides a builtin
% \ifCLASSINFOpdf conditional that works the same way.
% When switching from latex to pdflatex and vice-versa, the compiler may
% have to be run twice to clear warning/error messages.






% *** CITATION PACKAGES ***
%
%\usepackage{cite}
% cite.sty was written by Donald Arseneau
% V1.6 and later of IAENGtran pre-defines the format of the cite.sty package
% \cite{} output to follow that of IAENG. Loading the cite package will
% result in citation numbers being automatically sorted and properly
% "compressed/ranged". e.g., [1], [9], [2], [7], [5], [6] without using
% cite.sty will become [1], [2], [5]--[7], [9] using cite.sty. cite.sty's
% \cite will automatically add leading space, if needed. Use cite.sty's
% noadjust option (cite.sty V3.8 and later) if you want to turn this off.
% cite.sty is already installed on most LaTeX systems. Be sure and use
% version 4.0 (2003-05-27) and later if using hyperref.sty. cite.sty does
% not currently provide for hyperlinked citations.
% The latest version can be obtained at:
% http://www.ctan.org/tex-archive/macros/latex/contrib/cite/
% The documentation is contained in the cite.sty file itself.






% *** GRAPHICS RELATED PACKAGES ***
%

   \usepackage[outdir=./]{epstopdf}
   \usepackage[pdftex]{graphicx}
  % declare the path(s) where your graphic files are
  % \graphicspath{{../pdf/}{../jpeg/}}
  % and their extensions so you won't have to specify these with
  % every instance of \includegraphics
   \DeclareGraphicsExtensions{.pdf}

% graphicx was written by David Carlisle and Sebastian Rahtz. It is
% required if you want graphics, photos, etc. graphicx.sty is already
% installed on most LaTeX systems. The latest version and documentation can
% be obtained at:
% http://www.ctan.org/tex-archive/macros/latex/required/graphics/
% Another good source of documentation is "Using Imported Graphics in
% LaTeX2e" by Keith Reckdahl which can be found as epslatex.ps or
% epslatex.pdf at: http://www.ctan.org/tex-archive/info/
%
% latex, and pdflatex in dvi mode, support graphics in encapsulated
% postscript (.eps) format. pdflatex in pdf mode supports graphics
% in .pdf, .jpeg, .png and .mps (metapost) formats. Users should ensure
% that all non-photo figures use a vector format (.eps, .pdf, .mps) and
% not a bitmapped formats (.jpeg, .png). IAENG frowns on bitmapped formats
% which can result in "jaggedy"/blurry rendering of lines and letters as
% well as large increases in file sizes.
%
% You can find documentation about the pdfTeX application at:
% http://www.tug.org/applications/pdftex





% *** MATH PACKAGES ***
%
\usepackage[cmex10]{amsmath}
\usepackage{amsthm}
\usepackage{amssymb}

% A popular package from the American Mathematical Society that provides
% many useful and powerful commands for dealing with mathematics. If using
% it, be sure to load this package with the cmex10 option to ensure that
% only type 1 fonts will utilized at all point sizes. Without this option,
% it is possible that some math symbols, particularly those within
% footnotes, will be rendered in bitmap form which will result in a
% document that can not be IAENG compliant!
%
% Also, note that the amsmath package sets \interdisplaylinepenalty to 10000
% thus preventing page breaks from occurring within multiline equations. Use:
%\interdisplaylinepenalty=2500
% after loading amsmath to restore such page breaks as IAENGtran.cls normally
% does. amsmath.sty is already installed on most LaTeX systems. The latest
% version and documentation can be obtained at:
% http://www.ctan.org/tex-archive/macros/latex/required/amslatex/math/





% *** SPECIALIZED LIST PACKAGES ***
%
%\usepackage{algorithmic}
% algorithmic.sty was written by Peter Williams and Rogerio Brito.
% This package provides an algorithmic environment for describing algorithms.
% You can use the algorithmic environment in-text or within a figure
% environment to provide for a floating algorithm. Do NOT use the algorithm
% floating environment provided by algorithm.sty (by the same authors) or
% algorithm2e.sty (by Christophe Fiorio) as IAENG does not use dedicated
% algorithm float types and packages that provide these will not provide
% correct IAENG style captions. The latest version and documentation of
% algorithmic.sty can be obtained at:
% http://www.ctan.org/tex-archive/macros/latex/contrib/algorithms/
% There is also a support site at:
% http://algorithms.berlios.de/index.html
% Also of interest may be the (relatively newer and more customizable)
% algorithmicx.sty package by Szasz Janos:
% http://www.ctan.org/tex-archive/macros/latex/contrib/algorithmicx/




% *** ALIGNMENT PACKAGES ***
%
%\usepackage{array}
% Frank Mittelbach's and David Carlisle's array.sty patches and improves
% the standard LaTeX2e array and tabular environments to provide better
% appearance and additional user controls. As the default LaTeX2e table
% generation code is lacking to the point of almost being broken with
% respect to the quality of the end results, all users are strongly
% advised to use an enhanced (at the very least that provided by array.sty)
% set of table tools. array.sty is already installed on most systems. The
% latest version and documentation can be obtained at:
% http://www.ctan.org/tex-archive/macros/latex/required/tools/


%\usepackage{mdwmath}
%\usepackage{mdwtab}
% Also highly recommended is Mark Wooding's extremely powerful MDW tools,
% especially mdwmath.sty and mdwtab.sty which are used to format equations
% and tables, respectively. The MDWtools set is already installed on most
% LaTeX systems. The lastest version and documentation is available at:
% http://www.ctan.org/tex-archive/macros/latex/contrib/mdwtools/


% IAENGtran contains the IAENGeqnarray family of commands that can be used to
% generate multiline equations as well as matrices, tables, etc., of high
% quality.


%\usepackage{eqparbox}
% Also of notable interest is Scott Pakin's eqparbox package for creating
% (automatically sized) equal width boxes - aka "natural width parboxes".
% Available at:
% http://www.ctan.org/tex-archive/macros/latex/contrib/eqparbox/





% *** SUBFIGURE PACKAGES ***
%\usepackage[tight,footnotesize]{subfigure}
% subfigure.sty was written by Steven Douglas Cochran. This package makes it
% easy to put subfigures in your figures. e.g., "Figure 1a and 1b". For IAENG
% work, it is a good idea to load it with the tight package option to reduce
% the amount of white space around the subfigures. subfigure.sty is already
% installed on most LaTeX systems. The latest version and documentation can
% be obtained at:
% http://www.ctan.org/tex-archive/obsolete/macros/latex/contrib/subfigure/
% subfigure.sty has been superceeded by subfig.sty.



%\usepackage[caption=false]{caption}
%\usepackage[font=footnotesize]{subfig}
% subfig.sty, also written by Steven Douglas Cochran, is the modern
% replacement for subfigure.sty. However, subfig.sty requires and
% automatically loads Axel Sommerfeldt's caption.sty which will override
% IAENGtran.cls handling of captions and this will result in nonIAENG style
% figure/table captions. To prevent this problem, be sure and preload
% caption.sty with its "caption=false" package option. This is will preserve
% IAENGtran.cls handing of captions. Version 1.3 (2005/06/28) and later
% (recommended due to many improvements over 1.2) of subfig.sty supports
% the caption=false option directly:
%\usepackage[caption=false,font=footnotesize]{subfig}
%
% The latest version and documentation can be obtained at:
% http://www.ctan.org/tex-archive/macros/latex/contrib/subfig/
% The latest version and documentation of caption.sty can be obtained at:
% http://www.ctan.org/tex-archive/macros/latex/contrib/caption/




% *** FLOAT PACKAGES ***
%
%\usepackage{fixltx2e}
% fixltx2e, the successor to the earlier fix2col.sty, was written by
% Frank Mittelbach and David Carlisle. This package corrects a few problems
% in the LaTeX2e kernel, the most notable of which is that in current
% LaTeX2e releases, the ordering of single and double column floats is not
% guaranteed to be preserved. Thus, an unpatched LaTeX2e can allow a
% single column figure to be placed prior to an earlier double column
% figure. The latest version and documentation can be found at:
% http://www.ctan.org/tex-archive/macros/latex/base/



%\usepackage{stfloats}
% stfloats.sty was written by Sigitas Tolusis. This package gives LaTeX2e
% the ability to do double column floats at the bottom of the page as well
% as the top. (e.g., "\begin{figure*}[!b]" is not normally possible in
% LaTeX2e). It also provides a command:
%\fnbelowfloat
% to enable the placement of footnotes below bottom floats (the standard
% LaTeX2e kernel puts them above bottom floats). This is an invasive package
% which rewrites many portions of the LaTeX2e float routines. It may not work
% with other packages that modify the LaTeX2e float routines. The latest
% version and documentation can be obtained at:
% http://www.ctan.org/tex-archive/macros/latex/contrib/sttools/
% Documentation is contained in the stfloats.sty comments as well as in the
% presfull.pdf file. Do not use the stfloats baselinefloat ability as IAENG
% does not allow \baselineskip to stretch. Authors submitting work to the
% IAENG should note that IAENG rarely uses double column equations and
% that authors should try to avoid such use. Do not be tempted to use the
% cuted.sty or midfloat.sty packages (also by Sigitas Tolusis) as IAENG does
% not format its papers in such ways.


%\ifCLASSOPTIONcaptionsoff
%  \usepackage[nomarkers]{endfloat}
% \let\MYoriglatexcaption\caption
% \renewcommand{\caption}[2][\relax]{\MYoriglatexcaption[#2]{#2}}
%\fi
% endfloat.sty was written by James Darrell McCauley and Jeff Goldberg.
% This package may be useful when used in conjunction with IAENGtran.cls'
% captionsoff option. Some IAENG journals/societies require that submissions
% have lists of figures/tables at the end of the paper and that
% figures/tables without any captions are placed on a page by themselves at
% the end of the document. If needed, the draftcls IAENGtran class option or
% \CLASSINPUTbaselinestretch interface can be used to increase the line
% spacing as well. Be sure and use the nomarkers option of endfloat to
% prevent endfloat from "marking" where the figures would have been placed
% in the text. The two hack lines of code above are a slight modification of
% that suggested by in the endfloat docs (section 8.3.1) to ensure that
% the full captions always appear in the list of figures/tables - even if
% the user used the short optional argument of \caption[]{}.
% IAENG papers do not typically make use of \caption[]'s optional argument,
% so this should not be an issue. A similar trick can be used to disable
% captions of packages such as subfig.sty that lack options to turn off
% the subcaptions:
% For subfig.sty:
% \let\MYorigsubfloat\subfloat
% \renewcommand{\subfloat}[2][\relax]{\MYorigsubfloat[]{#2}}
% For subfigure.sty:
% \let\MYorigsubfigure\subfigure
% \renewcommand{\subfigure}[2][\relax]{\MYorigsubfigure[]{#2}}
% However, the above trick will not work if both optional arguments of
% the \subfloat/subfig command are used. Furthermore, there needs to be a
% description of each subfigure *somewhere* and endfloat does not add
% subfigure captions to its list of figures. Thus, the best approach is to
% avoid the use of subfigure captions (many IAENG journals avoid them anyway)
% and instead reference/explain all the subfigures within the main caption.
% The latest version of endfloat.sty and its documentation can obtained at:
% http://www.ctan.org/tex-archive/macros/latex/contrib/endfloat/
%
% The IAENGtran \ifCLASSOPTIONcaptionsoff conditional can also be used
% later in the document, say, to conditionally put the References on a
% page by themselves.





% *** PDF, URL AND HYPERLINK PACKAGES ***
%
%\usepackage{url}
% url.sty was written by Donald Arseneau. It provides better support for
% handling and breaking URLs. url.sty is already installed on most LaTeX
% systems. The latest version can be obtained at:
% http://www.ctan.org/tex-archive/macros/latex/contrib/misc/
% Read the url.sty source comments for usage information. Basically,
% \url{my_url_here}.





% *** Do not adjust lengths that control margins, column widths, etc. ***
% *** Do not use packages that alter fonts (such as pslatex).         ***
% There should be no need to do such things with IAENGtran.cls V1.6 and later.
% (Unless specifically asked to do so by the journal or conference you plan
% to submit to, of course. )


% correct bad hyphenation here
% \hyphenation{op-tical net-works semi-conduc-tor}


\begin{document}
%
% paper title
% can use linebreaks \\ within to get better formatting as desired
\title{Nondeterministic functional transducer inference algorithm}
%
%
% author names and IAENG memberships
% note positions of commas and nonbreaking spaces ( ~ ) LaTeX will not break
% a structure at a ~ so this keeps an author's name from being broken across
% two lines.
% use \thanks{} to gain access to the first footnote area
% a separate \thanks must be used for each paragraph as LaTeX2e's \thanks
% was not built to handle multiple paragraphs
%

\author{Aleksander Mendoza-Drosik}% <-this % stops a space


\newtheorem{theorem}{Theorem}
\newtheorem{definition}{Definition}


% note the % following the last \IAENGmembership and also \thanks -
% these prevent an unwanted space from occurring between the last author name
% and the end of the author line. i.e., if you had this:
%
% \author{....lastname \thanks{...} \thanks{...} }
%                     ^------------^------------^----Do not want these spaces!
%
% a space would be appended to the last name and could cause every name on that
% line to be shifted left slightly. This is one of those "LaTeX things". For
% instance, "\textbf{A} \textbf{B}" will typeset as "A B" not "AB". To get
% "AB" then you have to do: "\textbf{A}\textbf{B}"
% \thanks is no different in this regard, so shield the last } of each \thanks
% that ends a line with a % and do not let a space in before the next \thanks.
% Spaces after \IAENGmembership other than the last one are OK (and needed) as
% you are supposed to have spaces between the names. For what it is worth,
% this is a minor point as most people would not even notice if the said evil
% space somehow managed to creep in.



% The paper headers
%\markboth{}%
%{Shell \MakeLowercase{\textit{et al.}}:}
% The only time the second header will appear is for the odd numbered pages
% after the title page when using the twoside option.
%
% *** Note that you probably will NOT want to include the author's ***
% *** name in the headers of peer review papers.                   ***
% You can use \ifCLASSOPTIONpeerreview for conditional compilation here if
% you desire.




% If you want to put a publisher's ID mark on the page you can do it like
% this:
%\IAENGpubid{0000--0000/00\$00.00~\copyright~2007 IAENG}
% Remember, if you use this you must call \IAENGpubidadjcol in the second
% column for its text to clear the IAENGpubid mark.



% use for special paper notices
%\IAENGspecialpapernotice{(Invited Paper)}




% make the title area
\maketitle

\pagestyle{empty}
\thispagestyle{empty}


\begin{abstract}
%\boldmath

The purpose of this paper is to present an algorithm for inferring nondeterministic functional transducers. It has a lot in common with other well known algorithms such has RPNI and OSTIA. Indeed we will argue that this algorithm is a generalisation of both of them. Functional transducers are all those nondeterministic transducers whose regular relation is a function. Epsilon transitions as well as subsequential output can be erased for such machines, with the exception of output for empty string being lost. Learning partial functional transducers from negative examples is equivalent to learning total from positive-only data. 
\end{abstract}
% IAENGtran.cls defaults to using nonbold math in the Abstract.
% This preserves the distinction between vectors and scalars. However,
% if the journal you are submitting to favors bold math in the abstract,
% then you can use LaTeX's standard command \boldmath at the very start
% of the abstract to achieve this. Many IAENG journals frown on math
% in the abstract anyway.

% Note that keywords are not normally used for peerreview papers.
\begin{IAENGkeywords}
functional transducers, ostia, rpni, nondeterminism, grammatical inference
\end{IAENGkeywords}






% For peer review papers, you can put extra information on the cover
% page as needed:
% \ifCLASSOPTIONpeerreview
% \begin{center} \bfseries EDICS Category: 3-BBND \end{center}
% \fi
%
% For peerreview papers, this IAENGtran command inserts a page break and
% creates the second title. It will be ignored for other modes.
\IAENGpeerreviewmaketitle


\section{Introduction}
% The very first letter is a 2 line initial drop letter followed
% by the rest of the first word in caps.
%
% form to use if the first word consists of a single letter:
% \IAENGPARstart{A}{demo} file is ....
%
% form to use if you need the single drop letter followed by
% normal text (unknown if ever used by IAENG):
% \IAENGPARstart{A}{}demo file is ....
%
% Some journals put the first two words in caps:
% \IAENGPARstart{T}{his demo} file is ....
%
% Here we have the typical use of a "T" for an initial drop letter
% and "HIS" in caps to complete the first word.
\IAENGPARstart{L}{earning}
of nondeterministic automata has always been a topic of great interest, although not many positive results were achieved. Most of research focused on weighted automata\cite{DROSTE} and probabilistic machines\cite{MOHRI}\cite{MOHRI3}. Algorithms like APTI\cite{HANSAN} allowed for learning transducers from distribution. Some attempts at generalising non-probabilistic machines were also made, such as the semi-deterministic transducers\cite{semideterministic}. More results\cite{activeLearningNondetFST} were obtained by using active learning and queries. Relatively few research has been done that would attempt to learn non-deterministic automata from text only. In general case it can be proven that such a task is impossible. Only so far known positive results were for algorithms like OSTIA\cite{OSTIA}, RPNI\cite{RPNI} and its derivatives, but they assumed determinism. The algorithm in this paper presents a generalisation of the two previous algorithms that relaxes assumption of determinism. Here we only assume the transducer to be functional\cite{TRANSDUCERS}\cite{MendozaDrosik2020MultitapeAA}. 

\section{Assumptions}

The task is to learn functional nondeterministic transducers from informant. Their transition function is defined as $\delta: Q \times \Sigma \times Q \rightarrow \Gamma^*$ where $Q$ is set of states, $\Sigma$ is the input alphabet and $\Gamma$ is the output. All the formal relations that can be expressed by functional transducers are of the form $\Sigma^* \rightarrow \Gamma^*$. Because we do not allow subsequential transducers (only transitions have outputs but states don't) and neither allow for $\epsilon$-transitions, the empty string $\epsilon$ can only be in relation with another empty string $(\epsilon,\epsilon)$. Hence pairs of the form $(\epsilon,\gamma)$ for any non-empty $\gamma$ will never appear in informant . 



We can assume that the transducers are total for all non-empty strings (that is, the rational relation $\Sigma^* \rightarrow \Gamma^*$ is total, except for $(\epsilon,\epsilon)$, which may or may not belong to the relation), because every partial non-deterministic functional transducer can be reduced to a total one. Such reduction is done by adding some new special symbol $\#$ to the alphabet $\Gamma$ and creating a new total transducer that returns $\#$ for all inputs that would otherwise be rejected by the partial transducer. More precisely, it can be achieved by taking the input projection (accepted subset of $\Sigma^*$ for which partial transducer returns output), turning it into DFA, negating it and then turning it back into transducer by making it return $\#$ for all accepted inputs (except $\epsilon$). Lastly we need to perform union of this new negated transducer with the original one. Therefore it's possible to encode counterexamples only by using informant consisting of pairs $\Sigma^* \rightarrow \Gamma^* \cup \{\#\}$. Learning of partial transducers from negative examples is reducible to learning total transducers from only positive examples and vice versa. It's worth pointing out that such reduction would not be possible for deterministic transducers (due to preservation of prefixes).

We define informant as infinite sequence of pairs $\Sigma^* \times \Gamma^*$. Because our transducers are total, eventually every string from $\Sigma^+$ will appear in the informant. Because transducers are functional, every such  $\Sigma^*$ string uniquely determines $\Gamma^*$ output that appears along with its input in the informant. In order words, once we see pair $(\sigma,\gamma)$, we can be sure that next time we $(\sigma,\gamma')$, the outputs will be the same $\gamma=\gamma'$. During learning, the algorithm only has access to some finite initial segment of the informant, but we can make it as large as necessary. The learning will converge to some correct hypothesis in the limit as size of this segment approaches infinity.

Before showing the algorithm, let's first prove that learning in the limit is possible for functional non-deterministic transducers. This can be done by observing that finding the minimal transducer consistent with any finite part of informant is computable. We can enumerate all transducers starting from the small ones and slowing moving onto the larger ones until we eventually find one that returns expected outputs for all inputs. Now suppose there is some other transducer with no more states than the target transducer $T$ that we're trying to learn. If both transducers determine the same regular relations, then it doesn't matter which one we infer. However, if they are different, then we will at some point find a pair in the informant that tells them apart (transducer being functional, is the key here) and the inference algorithm will make a mind change. Because there are only finitely many automata smaller or equal to the target transducer, there will be only finite number of mind changes before reaching the correct hypothesis. Hence learning will always converge to some equivalent minimal transducer.

Such algorithm, is simple but not very practical. A polynomial procedure can be achieved by making one further restriction. We need to assume that the automata are not only functional but also unambiguous (that is, for any accepted input, there is only one possible accepting path). This restriction doesn't reduce the expressive power of automata, because every functional transducer can be converted to unambiguous one. The proof is simple and similar to powerset construction. Given some functional transducer $T$ with states $Q$,  build a new one $T'$ with set of states $Q' = Q \times 2^Q$. All transitions in original transducer $T$ are of the form $\delta(q_1,\sigma,q_2)=\gamma$ and $\delta$ is a partial function. We put a transition $\delta'(q_1',\sigma,q_2')=\gamma$ between $q_1' = (q_1,K_1)$  and $q_2' = (q_2,K_2)$  whenever $\delta(q_1,\sigma,q_2)=\gamma$ and $\hat{\delta}(K_1,\sigma)=K_2$, where $\hat{\delta}$ is the image of $\delta$ defined as $\hat{\delta}(K_1,\sigma)=\{q_2\in Q : \delta(q_1,\sigma,q_2)\ne\emptyset \}$. The state $q_1'=(q_1,K_1)$ is accepting whenever $q_1$ is accepting. At this point, the obtained powerset automaton is equivalent to the original one, but not unambiguous yet. The last step is to drop some of the transitions that are not necessary. If there are two transitions coming to the same $q_2'$ over the same symbol $\sigma$, they both must "carry" with them the same output (otherwise transducer wouldn't be functional). Hence one of them can be arbitrarily deleted. Analogically, if there are two states 
$q_1'=(q_1,K_1)$ and $q_2'=(q_2,K_2)$ such that $K_1=K_2$ and both $q_1$ and $q_2$ are accepting, then we don't need to make both $q_1'$ and $q_2'$ accepting. 

It's worth noting, that due to unambiguity, every element $(\sigma,\gamma)$ in the informant uniquely determines exactly one accepting path in the target transducer. However, it's not true that if $\sigma'$ is a prefix of $\sigma$, then path determined by $\sigma'$ is a prefix of path determined by $\sigma$.

\section{Initialization}

The inference algorithm needs to be initialized with maximal canonical prefix tree automaton, but due to nondeterminism, its construction is a little different from OSTIA or RPNI. Every state $q$ of the prefix tree corresponds to some state $\bar{q}$ of the original transducer $T$ that we are trying to learn (but the algorithm doesn't know $\bar{q}$). By  $\mathcal{L}(\bar{q})$ we will denote the relation defined by state $\bar{q}$, that is the relation that would be produced by $T$ if it's initial state was changed to $\bar{q}$. Note that even though $T$ defines a total relation, $\mathcal{L}(\bar{q})$ might be partial. By $\mathcal{L}(q)$ we denote the relation defined by state in the prefix tree automaton. In particular $\mathcal{L}(q_0)$ for initial state $q_0$ (root of the tree) is equal to the finite part of informant that we are using for learning.

For the purpose of the algorithm we also need the notion of Brzozowski's derivative but we extend it to regular relations. Given some pair of strings $(\sigma,\gamma)$ and some formal relation $L \subset \Sigma^* \times \Gamma^*$, we can take derivative $(\sigma,\gamma)^{-1}L$ defined as set of all strings in $L$ that begin with $(\sigma,\gamma)$, or more formally $\{(\sigma',\gamma') \in \Sigma^* \times \Gamma^* : (\sigma\sigma',\gamma\gamma') \in L \}$. We also need the $lcp$ function, which given some set of strings, returns their longest common prefix. Functions $\pi_\Sigma(L)$ and $\pi_\Gamma(L)$ are respectively input and output projections of formal relation $L$, which is formally defined as $\pi_\Sigma(L)=\{\sigma\in\Sigma^* : \exists_{\gamma\in\Gamma^*} (\sigma,\gamma)\in L \}$ (analogically for $\pi_\Gamma$).

The prefix tree $P$ is built recursively, starting from the root state. Before we begin the recursion we initialize $q_{i:o}=q_{\epsilon:\epsilon}$ as root of the tree. For any state $q_{i:o}$ of the prefix tree, we define $\mathcal{S}(q_{i:o})$ as the set of all strings in (some finite part of) informant, whose input starts with $i$ and output with $o$. Now we begin the recursion. We check if $(\epsilon,\epsilon)$ belongs to $\mathcal{S}(q_{i:o})$. If it does, we mark $q_{i:o}$ as accepting.  Next for every $\sigma\in\Gamma^*$, we check if there exists $(\sigma,a)$ in $\mathcal{S}(q_{i:o})$ where $a$ is any string $\Gamma^*$. If it does exist, then we create transition $(q_{i:o},\sigma,q_{i\sigma:oa},a)$ to some new state $q_{i\sigma:oa}$. For every symbol $\gamma$ from $\Gamma$, we take the derivative $D = (\sigma,\gamma)^{-1}\mathcal{S}(q_{i:o})$ and compute (if $D$ is not empty) longest common prefix of all possible outputs $lcp(\pi_\Gamma(D))=p$. Check if $oa$ is a prefix of $o\gamma p$ and if it's not, then we create transition $(q_{i:o},\sigma,q_{i\sigma:o\gamma p},\gamma p)\in\delta$ to some new state $q_{i\sigma:o\gamma p}$.  Note that $(\sigma,\gamma p)^{-1}\mathcal{S}(q_{i:o})=\mathcal{S}(q_{i\sigma:o\gamma p})$. By this point  $q_{i:o}$ may become a leaf (when $D$ was always empty), branch deterministically or non-deterministically. We need to perform the recursion for every outgoing branch. By the end of running this procedure we have $\mathcal{S}(q_{i:o})=\mathcal{L}(q_{i:o})$ for every state $q_{i:o}$ in the tree. It also guarantees that if two transitions come out of the same state over the same input symbol, then their the output of one is not a prefix of the other.  Moreover, if two transitions go out of the same state and have the same input symbol and start with the same output symbol, then exactly one of those transitions leads to accepting state.


We can show, that indeed for every valid informant there exists some $T$ such that for every state $q$ in the tree, there is a corresponding state $\bar{q}$ in $T$. The above procedure assumes that for any $\bar{q}\in Q$, $\sigma\in\Sigma$ and $\gamma\in\Gamma$, there cannot exist two distinct transitions $(\bar{q},\sigma,\bar{q_1},\gamma\gamma_1)$ and $(\bar{q},\sigma,\bar{q_2},\gamma\gamma_2)$, unless one of the states $\bar{q_1}$ or $\bar{q_2}$ is accepting. It also assumes that for any  $\bar{q}$ and $\sigma$, there cannot exist two transitions $(\bar{q},\sigma,\bar{q_1},\gamma_1)$ and $(\bar{q},\sigma,\bar{q_2},\gamma_2)$ such that $\gamma_1$ is a prefix of $\gamma_2$. To show that any transducer $T'$, can be converted to some equivalent $T$ that complies with those assumptions, suppose $Q'$ are the states of $T'$, then let $Q$ be the states of $T$ defined as $Q=2^{Q'\rightarrow\Gamma^*}$.
The conversion resembles powerset construction $2^{Q'}$, except that we additionally associate some string $\Gamma^*$ to every state of $Q'$ (that string will represent delayed transition output). 
We also enforce that in every state $\bar{q} \in Q$ (which is at the same time a subset $\bar{q} \subset Q'\rightarrow\Gamma^*$), the longest common prefix of all strings $\Gamma^*$ in $\bar{q}$ is $\epsilon$. For every state $\bar{q}\in Q$, let $U_{\bar{q}}$ be the set of all transitions from $T'$ that begin in some state $\bar{q}'$, which is a member of $\bar{q}$ and to the transition itself, we prepend the delayed output of $\bar{q}$. 
More formally if $(\bar{q}',\gamma')\in\bar{q}$ and $(\bar{q}',\sigma,\bar{q_2}',\gamma_2')\in\delta'$, then $(\bar{q}',\sigma,\bar{q_2}',\gamma'\gamma_2')$ belongs to $U_{\bar{q}}$. 
Now, for every $\bar{q}\in Q$ and $\sigma\in\Sigma$, check if there exists some transition  $(\bar{q}',\sigma,\bar{q_2}',a)$ in $U_{\bar{q}}$ such that $\bar{q_2}'$ is accepting (there can be at most one, if $T'$ is unambiguous). If it does, then we put transition in $T$ from $\bar{q}$ to state $\{(\bar{q_2}':\epsilon)\} \in Q$ over input $\sigma$ with output $a$. 
Next for every $\gamma\in\Gamma$, take all the transitions of $U_{\bar{q}}$,  whose input is $\sigma$ and output begins with $\gamma$ but also doesn't begin with $a$. 
Let $\gamma_2$ be the longest common prefix of all output strings of those transitions. 
(Note that $\gamma$ must be the first symbol of $\gamma_2$).  
Place a transition between $\bar{q}$ and $\bar{q_2}$ over input $\sigma$ with output $\gamma_2$, where $\bar{q_2}$ is defined as $\bar{q_2}=\{ (\bar{q_2}',\gamma_2^{-1}\gamma'\gamma_2') : \exists_{(\bar{q}',\gamma')\in\bar{q}} (\bar{q}',\sigma,\bar{q_2}',\gamma_2\gamma_2^{-1}\gamma'\gamma_2') \in U_{\bar{q}} \}$. In other words, $\bar{q_2}$ consists of all the states $Q'$ that are members of $\bar{q}$ and the transition from $\bar{q}$ to  $\bar{q_2}$ outputs the common prefix of all the relevant transitions from $T'$ , while the remaining suffixes is delayed and stored in $\bar{q_2}$ itself.
While, the set $2^{Q'\rightarrow\Gamma^*}$ is infinite (and thus, not suitable for defining transducer), if we assume $\{(\bar{q_0},\epsilon)\}$ to be the initial state of $T$ (where $\bar{q_0}$ is the initial state of $T'$), then only finitely many states of $Q$ are reachable. That's because all the possible strings $\Gamma^*$ that can be found in (reachable part of) $Q$ are bounded. Each time we create transition to $\bar{q_2}$, we set its output to longest common prefix $\gamma_2$, which in turn ensures that longest common prefix of all strings $\gamma_2^{-1}\gamma'\gamma_2'$ that are stored in $\bar{q_2}$ is the empty string $\epsilon$ (we wanted to enforce this property from the very beginning). It also means that $\gamma_2^{-1}$ fully consumes $\gamma'$, so the string stored in $\bar{q_2}$ is either equal to or is some suffix of $\gamma_2'$. For this reason the strings stored in $2^{Q'\rightarrow\Gamma^*}$  do not "accumulate" over time. The delayed strings can only be suffixes of some existing transition outputs found in $T'$.  Moreover, notice that if the transducer is unambiguous, the $Q$ is indeed $2^{Q'\rightarrow\Gamma^*}$ rather than $2^{Q'\times\Gamma^*}$, because there cannot exist two states $\bar{q}'$ that both transition to the same $\bar{q_2}'$. The accepting states of $Q$ are marked accordingly, whenever they contain some (exactly one or zero) accepting states of $Q'$.  This concludes the conversion. One should notice that this construction preserves unambiguity of transducer. 

We can also show, that for every state $q_2$ in the prefix tree transducer, its incoming edge $e=(q_1,\sigma,q_2,\gamma)$ exactly corresponds to the same edge $\bar{e}=(\bar{q_1},\sigma,\bar{q_2},\gamma)$ in $T$, as soon as all the outgoing edges of $q_2$ have been discovered from the informant. Note that, if some outgoing edges of $q_2$ were missing (not yet known), then the prefix tree transducer might go "too far" in onward form. More precisely, the longest common prefix of all outputs $\Gamma^*$ of all outgoing transitions of $\bar{q_2}$ must be equal to $\epsilon$, but if some of the outgoing transitions of $q_2$ were missing, then their longest common prefix might be a non-empty string $\gamma_2\ne\epsilon$, and it would then be pushed onward to $e=(q_1,\sigma,q_2,\gamma\gamma_2)$. 

We can use the result above to show that for any state $q_{i:o}$ all the outgoing edges of $q_{i:o}$ will be discovered as soon as we read all of the strings $i\Sigma^{\le m}$ where $m$ is the size of target transducer $T$. Assuming that $T$ is trim, every outgoing edge of $q_{i:o}$ will eventually lead to some accepting state. The length of this accepting path can be at most $m$, becasue if it was longer, then by pigeon-hole principle some state would need to repeat and we could find a shorter path without the repetition. Hence if we read all $\Sigma^{\le 2n}$ strings, then we can be sure that all states $q_{i:o}$ with $i<m$ have all of their outgoing edges discovered and their outputs are exactly the same as those of the corresponding edges in $T$.



\section{Inference algorithm}


Inference algorithm is similar to RPNI and OSTIA. We attempt to merge states and look for arising ambiguous paths. Every two ambiguous paths must be unified until all ambiguity is eliminated. The unification relies on pushing-back outputs whenever necessary. Paths that return different outputs (and thus, break assumption of functional transducer) cannot be unified.  Similarly push-backs that are not transduction-preserving (i.e. a push-back that changes regular relation recognized by transducer) will fail. Merging is also rejected when one ambiguous path contains one of the states that we are trying to merge, but the other one doesn't. If none of the above scenarios occur, and all ambiguous paths are unified, then merge is accepted and inference progresses. The order in which merges are attempted is very important and must be breath-first, or otherwise learning in the limit won't be guaranteed. Below we provide more details.


First we fix an order among states of $P$, such that $q_{i_1:o_1}<q_{i_2:o_2}$ whenever $i_1<_{lex-len}i_2$, where $lex-len$ stands for length-lexicographic order, such that shorter strings are lesser than the longer ones. Now there are two loops in our algorithm. The outer loop iterates all states $q_{i_2:o_2}$ of $P$ and the inner loop iterates only those states $q_{i_1:o_1}$ that already came earlier, that is $q_{i_1:o_1}<q_{i_2:o_2}$. Both loops iterate in increasing $<_{lex-len}$ order. For every pair,of states we attempt to perform their merge. To do this we need to detect ambiguous paths. If merging succeeds, the end result is the deletion of $q_{i_2:o_2}$. The $q_{i_1:o_1}$ retains both states' transitions. 

Checking whether automaton is ambiguous can be done in quadratic time by the squaring procedure\cite{Marie-Pierre}. In order to find the exact paths that are ambiguous, the procedure can be extended to work like a graph search. Squaring of automaton is nothing more than taking its cross product with itself. If $Q$ are states of transducer $P$, then $Q\times Q$ are states of squared automaton $P\times P$. If transducer $P$ has two transitions $(q_1,\sigma,q_2,\gamma_2)$ and $(q_1',\sigma,q_2',\gamma_2')$, then we put a transition $((q_1,q_1'),\sigma,(q_2,q_2'))$ in the squared automaton $P\times P$ (Notice that we lose track of outputs. They are not needed for our purposes). If we  assume that $P$ is trim (all states are reachable and no state is a dead-end) and at any point we encounter a pair $(q_1,q_2)$ in $P\times P$ such that both $q_1$ and $q_2$ transition over the same $\sigma$  to either the same state $q_3$ (formally, there is a transition $((q_1,q_2),\sigma,(q_3,q_3))$ in $P\times P$) or two different accepting states $q_3$ and $q_4$, then we can conclude that $P$ is ambiguous. Hence, finding an ambiguous path reduces to implementing a path-finding algorithm that searches the graph of $P\times P$ for a pair of states $(q_1,q_2)$. 

For additional optimisation, the search can be done incrementally. First we collect all reachable states of $P \times P$. Then as we merge $q_{i_1:o_1}$ with $q_{i_2:o_2}$, only don't "physically merge" them. Instead we scan the set of already reached pairs, and whenever we see $(q_{i_1:o_1},q_2)$ we add $(q_2,q_{i_2:o_2})$ to the set. Similarly if we see $(q_1,q_{i_2:o_2})$ we add $(q_{i_1:o_1},q_1)$. We can also treat the pairs as unordered and this way we don't need to check  $(q_{i_1:o_1},q_1)$ and  $(q_1,q_{i_1:o_1})$ twice. Once we added all those pairs to the set of reachable pairs, we rerun the path-finding procedure and try to discover more reachable states in $P \times P$. 


Suppose that the above procedure detected two ambiguous accepting paths 
\begin{equation*}
\begin{split}
(q_0,\sigma_1,q_1,\gamma_1),(q_1,\sigma_2,q_2,\gamma_2),
...(q_{n-1},\sigma_n,q_n,\gamma_n) \\
(q_0,\sigma_1,q_1',\gamma_1'), (q_1',\sigma_2,q_2',\gamma_2'),... (q_{n-1}',\sigma_n,q_n',\gamma_n')
\end{split}
\end{equation*}
that both start in initial state $q_0$. All ambiguous paths must start in $q_0$, because the ambiguity detection procedure allows for "jumps" from $q_{i_1:o_1}$ to $q_{i_2:o_2}$ and vice versa. 
Now in order to ensure that the automaton is unambiguous, we need to merge all the states: $q_1$ with $q_1'$, $q_2$ with $q_2'$, $q_3$ with $q_3'$ and so on. Whenever we encounter $q_{i_1:o_1}=q_k$ or $q_{i_2:o_2}=q_k$ for some $k\le n$ we have to make sure that $q_{i_2:o_2}=q_k'$ or $q_{i_1:o_1}=q_k'$. If that is not the case, then merging must fail immediately. 


If all pairs of $q_k$ and $q_k'$ satisfy the above constraint, then we can start unification procedure. The prefix tree automaton $P$ was built in such a way that all transitions are in onward form, but now we might need to push back some of the outputs. We need to ensure $\gamma_1=\gamma_1'$, $\gamma_2=\gamma_2'$... $\gamma_n=\gamma_n'$. Only push back operations are allowed, that is, we can "cut off" the suffix from $\gamma_k$ and  prepend it to $\gamma_{k+1}$, but we are not allowed to "cut off" prefix of $\gamma_{k+1}$ and append it to $\gamma_k$. Similarly for $\gamma_k'$ and $\gamma_{k+1}' $. Attention must be paid as some push-backs might alter the regular relation recognized by transducer. For simplicity, we ensure that all push-backs are transduction-preserving  by only allowing them, when state has only one incoming transition. That is a suffix of $\gamma_k$ can become a prefix of $\gamma_{k+1}$ under the condition that state $q_k$ has one single incoming transition $(q_{i-1},\sigma_k,q_k,\gamma_k)$. Therefore, sometimes path unification may fail and some states may not be merged.
State merging might also fail if it breaks functionality of transducer, that is, if it there are two accepting paths but their outputs are different $\gamma_1\gamma_2...\gamma_n\ne\gamma_1'\gamma_2'...\gamma_n'$. 



\section{Proof of learning in the limit}


Let $P$ be the prefix tree transducer. We assume that the target transducer $T$ is functional, unambiguous, trim, total (except for $\epsilon$) and its transitions satisfy the following properties: 


Suppose that $T$ has $m$ states and that we saw all of the inputs $\Sigma^{\le 2m}$ from informant. Hence for every $q_{i_1:o_1}$ in $P$ such that $\vert i_1\vert < m$,  if there exists an edge $(\bar{q}_{i_1:o_1},\sigma,q_2,\gamma)$ in $T$, then there must exist $(q_{i_1:o_1},\sigma,q_{i_1\sigma:o_1\gamma},\gamma)$ in $P$ such that $\bar{q}_{i_1\sigma:o_1\gamma}=q_2$. We will refer to every transition  $(q_{i_1:o_1},\sigma,q_{i_1\sigma:o_1\gamma},\gamma)$ in $P$ using the unique label $e_{i_1\sigma:o_1\gamma}$. The above observation can be restated as for every $e_{i_1\sigma:o_1\gamma}$ such that $\vert i_1\sigma\vert \le m$ there exists $\bar{e}_{i_1\sigma:o_1\gamma}$ in $T$ whose output is exactly $\gamma$ and target state of $e_{i_1\sigma:o_1\gamma}$ corresponds to target state of $\bar{e}_{i_1\sigma:o_1\gamma}$.


When we merge two states $q_{i_1:o_1}$ and $q_{i_2:o_2}$ in $P$ such that $\vert i_1 \vert < m$ and $\vert i_2 \vert < m$, there might arise many ambiguous paths and many transitions need to be unified. Suppose that $\bar{q}_{i_1:o_1}=\bar{q}_{i_2:o_2}$ and we unify $e_{i_3:o_3}$ with $e_{i_4:o_4}$. There are four cases. 
\begin{enumerate}
	\item Suppose $\vert i_3 \vert \le m $ and  $\vert i_4 \vert \le m $ then their outputs must be exactly equal and no push-back is necessary.
	\item Suppose $\vert i_3 \vert \le m $ and  $\vert i_4 \vert > m $ then the output of $e_{i_4:o_4}$ might require to be pushed-back and as a result it will become equal to output of $e_{i_3:o_3}$ and $\bar{e}_{i_3:o_3}$. Hence $e_{i_4:o_4}$ will never need to be pushed-back again.
	\item  $\vert i_3 \vert > m $ and  $\vert i_4 \vert \le  m $ is same as above
	\item Suppose $\vert i_3 \vert > m $ and  $\vert i_4 \vert > m $ then push-backs might be necessary on either side and they are not guaranteed to result the exact same output as in $T$. However, it is guaranteed that $e_{i_3:o_3}$ is the only transition incoming to its target state and similarly is $e_{i_4:o_4}$ (because inference algorithm will never attempt to merge targets of two such edges). Hence after we unify target of $e_{i_3:o_3}$ with target of $e_{i_4:o_4}$, the resulting state will also have only one incoming transition. Therefore all future push-backs on this transition will be transduction-preserving and eventually the correct transition output will be inferred. 
\end{enumerate}

Because transducer is total, this proves that if we are able to correctly identify and merge all those states $q_{i_1:o_1}$ and $q_{i_2:o_2}$ with $\vert i_1 \vert < m$ and $\vert i_2 \vert < m$, then the transducer will be fully inferred before we have the chance to merge any states further than $m$.

In order to prove that only and all the correct merges will be performed, we first prove that $q_{i_1:o_1}$ will be merged with $q_{i_2:o_2}$ only if $\mathcal{L}(\bar{q}_{i_1:o_1})=\mathcal{L}(\bar{q}_{i_2:o_2})$ and that no correct merge will be mistakenly missed. There are the following cases.

\begin{enumerate}
	\item Suppose that $\mathcal{L}(\bar{q}_{i_1:o_1}) \cup \mathcal{L}(\bar{q}_{i_2:o_2})$ is not a function.
	Then there exists some string $i\in\Sigma^*$ such that $(i,o) \in \mathcal{L}(\bar{q}_{i_1:o_1})$ and $(i,o') \in \mathcal{L}(\bar{q}_{i_2:o_2})$ and that $o\ne o'$. We need to wait until informant shows us the two examples $(i_1 i, o_1 o)$  and $(i_2 i, o_2 o')$ and then merging $q_{i_1:o_1}$ with $q_{i_2:o_2}$ will become impossible, because the two paths will be ambiguous and have outputs impossible to unify. 
	
	\item Suppose that $\mathcal{L}(\bar{q}_{i_1:o_1}) \subset \mathcal{L}(\bar{q}_{i_2:o_2})$ holds and $\mathcal{L}(\bar{q}_{i_2:o_2}) \subset \mathcal{L}(\bar{q}_{i_1:o_1})$ doesn't. Then there will exist some $(i,o')\in \mathcal{L}(\bar{q}_{i_2:o_2})$ such that 
	$i$ is not present in $\mathcal{L}(\bar{q}_{i_1:o_1})$.	
	Because transducer is total, there must exist (in the limit) some state $q_{i_1:o_3}$ such that $(i,o)\in \mathcal{L}(q_{i_1:o_3})$. If we attempt to merge   $q_{i_1:o_1}$ with $q_{i_2:o_2}$ we will detect two ambiguous paths $(i_1 i, o_3 o')$ and $(i_1 i , o_3 o)$. Those paths will be respectively 

	\begin{equation*}
	\begin{split}
	(q_0,\sigma_1,q_1,\gamma_1),...
	(q_{k-1},\sigma_k,q_{i_2:o_2},\gamma_k),...
	(q_{n-1},\sigma_n,q_n,\gamma_n) \\
	(q_0,\sigma_1,q_1',\gamma_1'),... (q_{k-1}',\sigma_k,q_{i_1:o_3},\gamma_k'),... (q_{n-1}',\sigma_n,q_n',\gamma_n')
	\end{split}
	\end{equation*}

	and the inputs are $i_1=\sigma_1\sigma_2...\sigma_k$ and $i=\sigma_{k+1}...\sigma_n$. You can see that $q_k=q_{i_2:o_2}$ but $q_k'$ is neither $q_{i_1:o_1}$ nor $q_{i_2:o_2}$, hence merging will be rejected.
	
	\item Suppose that $\mathcal{L}(\bar{q}_{i_2:o_2}) \subset \mathcal{L}(\bar{q}_{i_1:o_1})$ holds and  $\mathcal{L}(\bar{q}_{i_1:o_1}) \subset \mathcal{L}(\bar{q}_{i_2:o_2})$ doesn't, then do analogically as above.
\end{enumerate}

Hence we proved that, in the limit, merges will be done only if $\mathcal{L}(\bar{q}_{i_1:o_1})=\mathcal{L}(\bar{q}_{i_2:o_2})$. Now we need to prove that no correct merges will be missed. A merge can be rejected for 3 reasons:

\begin{enumerate}
	\item Path cannot be unified, because some push-back is not transduction-preserving. We already proved above that this is not an issue.
	\item Path cannot be unified, because the outputs are different and break functionality. If this happens, then the merge cannot be correct.
	\item Path cannot be unified because for some $k$ the state $q_k$ is either  $q_{i_1:o_1}$ or $q_{i_2:o_2}$ but the state  $q_k'$ is neither of those. Let's prove that this can never be a correct merge.
	
	We define configuration $K$ as any subset of $Q \rightarrow \Gamma^*$. By $K_i$ we denote the configuration reached after reading input $i\in\Sigma^*$. Formally $K_\epsilon$ is the singleton set $\{(q_0,\epsilon)\}$ and recursive definition of $K_{i \sigma}$ is $\{(q_2,o\gamma) \in  Q \rightarrow \Gamma^* : \exists_{(q_1,o)\in K_i} (q_1,\sigma,q_2,\gamma)\in\delta \}$. 
	
	If any two states $q$ and $q'$ in $P$ such that $\bar{q}=\bar{q}'$,  belong to $K_i$ for some $i\in\Sigma^*$,  then $q=q'$, because otherwise there would exist two different paths over $i$ in $T$ that both lead to $\bar{q}$ and $T$ would not be functional (and transitions of $P$ guarantee us that if there are two different paths over $i$ then they have distinct outputs that are not prefixes of one another). This guarantees us that as we attempt to merge $q_{i_1:o_1}$ with $q_{i_2:o_2}$, then for any $K_i$, the $q_{i_1:o_1}$ is in $K_i$ if and only if $q_{i_2:o_2}$ is in $K_i$ and those are the only states in $K_i$ that correspond to $\bar{q}_{i_1:o_1}$. 
	
\end{enumerate}

To finish the proof we need to conclude that $\mathcal{L}(\bar{q}_{i_1:o_1})=\mathcal{L}(\bar{q}_{i_2:o_2})$ implies $\bar{q}_{i_1:o_1})=\bar{q}_{i_2:o_2}$. This holds true because, if it didn't we could find another transducer equivalent to $T$, but smaller, by merging $\bar{q}_{i_1:o_1}$ with $\bar{q}_{i_2:o_2}$.






\iffalse
\section{Optimisation}

It's possible to make inference converge much faster and make it attempt merging of relatively few states. Instead of merging in breath-first order, we use $\mathcal{S}$ determine the what states are most adequate for merging.  More precisely, suppose there are two states $q_1$ and $1_2$. We should count cardinality of the intersection $\mathcal{S}(q_1) \cap \mathcal{S}(q_2)$. We will refer to this number as $\phi(q_1,q_2)$. We should always merge the pair with the most overlap (largest possible $\phi(q_1,q_2)$ value) first, before merging any other, less overlapping pairs. Every time we merge $q_1$ with $q_2$ and produce a new merged state $q_3$, we update $\mathcal{S}$ accordingly as $\mathcal{S}(q_3)=\mathcal{S}(q_1) \cup \mathcal{S}(q_2)$. Naturally, the states that should be merged will have more outgoing paths overlapping, than those that come from different original states $\bar{q_1}\ne\bar{q_2}$. This is a powerful heuristic, but it has one disadvantage. It violates learning in the limit. Instead it guarantees learning in PAC framework. As the size of informant increases, the probability that the wrong transducer will be inferred approaches zero. The proof is as follows.

There is no issue when $\bar{q_1}$ and $\bar{q_2}$ are both equal and can be safely merged, therefore let's focus on the interesting case and
suppose that $\bar{q_1}\ne\bar{q_2}$.  If the intersection $\mathcal{L}(\bar{q_1}) \cap \mathcal{L}(\bar{q_2})$ is finite, then the algorithm might merge them incorrectly for only finite parts of informant, but in the limit it will always find some counterexample that tells them apart. If the intersection is infinite, then (according to pumping lemma) there must be some overlapping path starting in $q_1$ going through states $q_1'$ and $q_1''$ such that $\bar{q_1'}=\bar{q_1''}$. Similarly for $q_2$, $q_2'$ and $q_2''$. If $\bar{q_1'}=\bar{q_2'}$ then they are safe to be merged and we only need to wait for the counterexample distinguishing $q_1$ from $q_2$. If  $\bar{q_1'}\ne\bar{q_2'}$ then even after finding counterexample distinguishing $q_1'$ and $q_2'$, there might appear many more common strings that favour merging of $q_1''$ and $q_2''$. This problem might repeat forever for $q_1'''$, $q_1''''$ and so on, but we can observe that $\phi(q_1',q_1'') > \phi(q_1'',q_2'')$. (Note that $\mathcal{S}(q_1)$ is always finite, hence $phi(q_1,q_2)$ is finite as well.  During the inference we merge some states and introduce cycles, which may cause $\mathcal{L}(q_1)$ to become infinite, and then two inifnite cardinalities could not be compared. This does not affect $\mathcal{S}$. After we merge states $q_1$ and $q_2$ into one new state $q_3$, the value of $\mathcal{S}(q_3)$ is updated, but for all other states,  $\mathcal{S}$ remains untouched. Hence comparison of $\phi$ values is always well defined.) That's because all those common strings that make up $ \phi(q_1'',q_2'')$ also contribute to $\phi(q_1',q_1'')$. Additionally all the counterexamples that tell $q_1''$ apart from $q_2''$, $q_1'''$ apart from $q_2'''$ and so on, also contribute to $\phi(q_1',q_1'')$. The same goes for $\phi(q_2',q_2'')$. Hence all infinite repeating cycles will be merged correctly. If we first merge all those states that should be merged and are no further from the root than some finite number $n$, then the automaton will be fully inferred before it gets the chance to make any remaining incorrect merges. In all those cases above we can see, that in the limit, incorrect merges will either have lower values of $\phi$ than the correct ones, or will be rejected by counterexamples. The only problem is that even though the pairs of states close to the root (both states in the pair are no further than $n$) will be merged correctly, we don't have any guarantee about pairs $q_1$ and $q_2$  where $q_1$ is close to the root and  $q_2$ is far from it (further than $n$). In the pessimistic scenario, informant will first show us many examples that increase $\phi(q_1,q_2)$, until it becomes sufficiently large to be merged first, and then shows us a counterexample. And this cycle could repeat infinitely, hence inference algorithm would oscillate between correct hypothesis and a wrong one. This breaks learning in the limit. However, if we consider Martin-Löf's definition of random strings, we can notice that the set of malicious informants that trick learning to infer wrong hypotheses is a non-measurable set. Hence such informants are not Martin-Löf random. 

\fi

\section{Conclusions}
This concludes the description of onward functional transducer inference algorithm. There is one interesting thing we would like to point out. One could say that RPNI is a special case of OSTIA, where the output is always the empty string. In particular it should be observed that every finite state automaton is a special case of finite state transducer that either rejects (prints null $\emptyset$ output) or accepts (prints empty $\epsilon$ output). The algorithm described in this paper is a "superalgorithm" that can behave like OSTIA when the target transducer is deterministic (and outputs preserve their prefixes). 
Moreover, note that there is no need to introduce subsequential transducers, because the state output can be simulated with non-determinism. The only exception being the state output of initial state. This limitation is not a problem, because the output generated by initial state (output associated with empty input string), can be learned independently. More precisely, as soon as informant shows us the output associated with $\epsilon$, we save it somewhere aside and then learn the transducer as usual, by pretending that $(\epsilon,\epsilon)$ belongs to the regular relation.


% Can use something like this to put references on a page
% by themselves when using endfloat and the captionsoff option.
\ifCLASSOPTIONcaptionsoff
  \newpage
\fi



% trigger a \newpage just before the given reference
% number - used to balance the columns on the last page
% adjust value as needed - may need to be readjusted if
% the document is modified later
%\IAENGtriggeratref{8}
% The "triggered" command can be changed if desired:
%\IAENGtriggercmd{\enlargethispage{-5in}}

% references section

% can use a bibliography generated by BibTeX as a .bbl file
% BibTeX documentation can be easily obtained at:
% http://www.ctan.org/tex-archive/biblio/bibtex/contrib/doc/
% The IAENGtran BibTeX style support page is at:
% http://www.michaelshell.org/tex/IAENGtran/bibtex/
%\bibliographystyle{IAENGtran}
% argument is your BibTeX string definitions and bibliography database(s)
%\bibliography{IAENGabrv,../bib/paper}
%
% <OR> manually copy in the resultant .bbl file
% set second argument of \begin to the number of references
% (used to reserve space for the reference number labels box)
% \begin{thebibliography}{1}

% \bibitem{IJCS}
% N.~Meghanathan and G.~W. Skelton, ``Risk Notification Message
% Dissemination Protocol for Energy Efficient Broadcast in Vehicular
% Ad hoc Networks,'' {\it IAENG International Journal of Computer
% Science}, vol. 37, no. 1, pp. 1-10, Jul. 2010.

% \end{thebibliography}


% >>>>>>>>>>>>>>>>>>>>>> Bibliography <<<<<<<<<<<<<<<<<<<<<<<<<<<<<<<<<<<<<

\bibliographystyle{BibTeXtran}   % (uses file "BibTeXtran.bst")
\bibliography{BibTeXrefs}       % (expects the reference in the file "BibTeXrefs.bib")



% biography section
%
% If you have an EPS/PDF photo (graphicx package needed) extra braces are
% needed around the contents of the optional argument to biography to prevent
% the LaTeX parser from getting confused when it sees the complicated
% \includegraphics command within an optional argument. (You could create
% your own custom macro containing the \includegraphics command to make things
% simpler here.)
%\begin{biography}[{\includegraphics[width=1in,height=1.25in,clip,keepaspectratio]{mshell}}]{Michael Shell}
% or if you just want to reserve a space for a photo:

%\begin{IAENGbiography}{Michael Shell} Biography text here.
%\end{IAENGbiography}

% insert where needed to balance the two columns on the last page with
% biographies
%\newpage

%\begin{IAENGbiographynophoto}{Jane Doe}
%Biography text here.
%\end{IAENGbiographynophoto}

% You can push biographies down or up by placing
% a \vfill before or after them. The appropriate
% use of \vfill depends on what kind of text is
% on the last page and whether or not the columns
% are being equalized.

%\vfill

% Can be used to pull up biographies so that the bottom of the last one
% is flush with the other column.
%\enlargethispage{-5in}



% that's all folks
\end{document}
